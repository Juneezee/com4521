\documentclass[12pt, a4paper]{article}
\edef\restoreparindent{\parindent=\the\parindent\relax}
\usepackage{amsmath}
\usepackage[UKenglish]{babel}
\usepackage[bibstyle=ieee, dashed=false, sorting=nty]{biblatex}
\usepackage[labelfont=bf]{caption}
\usepackage{colortbl}
\usepackage{csquotes}
\usepackage{fancyhdr}
\usepackage{float}
\usepackage[top=25mm, right=25mm, bottom=25mm, left=25mm]{geometry}
\usepackage{graphicx}
\usepackage[hidelinks]{hyperref}
\usepackage{listings}
\usepackage{longtable}
\usepackage{minted}
\usepackage{microtype}
\usepackage{multicol}
\usepackage{parskip}
\usepackage{pgfplots, tikz}
\usepackage{tcolorbox}
\usepackage{xcolor}
\usepackage{textcomp} % for trademark, copyright symbol

\restoreparindent

\pagestyle{fancy}
\fancyhead[L]{COM4521}
\fancyhead[C]{Assignment: N-body Simulation}
\fancyhead[R]{Part 1}
\setlength{\headheight}{15pt}

\usemintedstyle{vs}

\addbibresource{references.bib}

\begin{document}

\section{Pre-Implementation Considerations}
Several aspects are considered in advance to facilitate the implementation process and to improve
the overall programming quality.

\subsection{Programming Practices} \label{sec:practices}
To minimise the risk of programming pitfalls, a set of good practices is adopted. These practices
are assessed according to their relative importance to the quality. Below is the list of practices
ordered in descending importance level.

\begin{enumerate}
  \item \textbf{Robustness:} The program must be able to handle any user input without crashing. If
  a given input is invalid, the program should return a descriptive error message.
  \item \textbf{Maintainability:} Follow a consistent coding style and use the self-documenting
  function and variable names. Refactor the code so that each function is only responsible for one
  task. Code with good structure and high cohesion will be able to narrow down the range of
  potential problems in the future.
  \item \textbf{Computational Resource Efficient:} Perform the minimum amount of computation in a
  loop if possible. Minimise overhead by reducing the number of function calls in a loop. If the
  value of a computationally expensive operation will be used multiple times, then store it into a
  variable.
  \item \textbf{Documentation:} The code should be well commented. Each function should have a
  Javadoc style block comment before the function definition. It is preferred to write
  self-documenting code, however, inline comments should be used when further clarification is
  required.
\end{enumerate}

\subsection{Unit Testing}
Since the program involves user input processing, thus it is required to perform basic unit testing
to test its correctness and completeness in handling user inputs.

It is impractical to test every possible combination of user inputs. Therefore, each command
argument will be tested independently with their respective edge cases. Testing the program with
edge cases is an efficient way to ensure that the problem can handle extreme user inputs.

\begin{table}[H]
  \centering
  \begin{tabular}{|c|c|c|c|}
    \hline \rowcolor[HTML]{D8D8D8}
    Test Case & Argument & Expected Result & Status \\ \hline
    & & & \\ \hline
    \end{tabular}
  \caption{Documentation format of unit testing.}
\end{table}

\subsubsection{Limitations}
Unit testing with edge cases will not catch every error in the program, such as runtime errors or
system-level errors. It is possible for the program to produce incorrect behaviour while passing all
the test cases. To mitigate this issue, the number of benchmarking has to be increased to reduce the
likelihood of unexpected errors.

\section{Preliminary Tasks Implementation}
This section discusses and compares the various possible techniques for implementing preliminary
tasks. Justifications are given for any final decision or approach taken.

\subsection{Arguments Processing}
\subsubsection{Integer Arguments} \label{subsec:args}
\begin{tcolorbox}
\textbf{Problem:} The integer arguments (\textbf{N}, \textbf{D}, and \textbf{I}) are saved as
`string' in \mintinline{c}{char *argv[]}.

\bigskip\textbf{Goal:} Parse and convert the `string' into \mintinline{c}{unsigned int} (the same
data type used in \mintinline{c}{NBodyVisualiser.c}).
\end{tcolorbox}

Each input argument \textbf{N}, \textbf{D}, and \textbf{I} can be categorised as follows:
\begin{enumerate}
    \item \textbf{Category 1:} Argument is a string (e.g. "abc"). In this case, the program must
    exit with an error message since the argument is not a valid integer and cannot be converted.
    \item \textbf{Category 2:} Argument is a negative number. Since the data type will be
    \mintinline{c}{unsigned int}, the program must exit with an error message.
    \item \textbf{Category 3:} Argument is zero. For \textbf{N} and \textbf{D}, the value must be
    larger than 1 so the program will exit with an error message. For \textbf{I}, the program will
    accept the argument but start in \textbf{visualisation} mode.
    \item \textbf{Category 4:} Argument is a positive number. In this category, there are two more
    cases to be considered:
    \begin{enumerate}
        \item argument $\leq$ \mintinline{c}{INT_MAX}. The program should accept the argument and
        produce no error. If the number is very large and the machine has limited computational
        resources, then the program will either return a memory allocation error (discussed in
        \hyperref[sec:malloc]{\textbf{Section \ref*{sec:malloc}}}) before the simulation or
        encounter errors including out-of-memory error, system not responding error, etc. during the
        simulation.
        \item argument $>$ \mintinline{c}{INT_MAX}. The is the case where overflow occurs. The
        program should return an error and exit.
    \end{enumerate}
\end{enumerate}

\textcolor{red}{N.B.} Changes are made to the definition of \textbf{Category 4} in second iteration
of implementation. Refer to \hyperref[subsec:maxd]{\textbf{Section \ref*{subsec:maxd}}} for detailed
explanation.

Based on the analysis of input categorisation, the problem is now clearly defined and the
implementation approach can be chosen. The C standard library \mintinline{c}{stdlib.h} provides
different functions that can convert a string to different integer types. Firstly, the
implementation has to choose a function with appropriate return type. However, there are several
return types that are not suitable for the use case, as justified below:
\begin{enumerate}
    \item A function with return type of \mintinline{c}{int} is not able to
    satisfy the conditions in \textbf{Category 4}. Argument inputs that are larger than
    \mintinline{c}{INT_MAX} will cause overflow of \mintinline{c}{int}.

    \item According to the official Microsoft documentation on storage of basic types
    \cite{basic_types_storage}, \mintinline{c}{long} has the same storage size as
    \mintinline{c}{int} (4 bytes). Therefore, it will have the same issue as (1).

    \item Function that returns \mintinline{c}{float}, \mintinline{c}{double}, or
    \mintinline{c}{long double} are not suitable as they are primarily use for floating point
    numbers.
\end{enumerate}

After applying these restrictions, the only functions left for selection are:
\begin{itemize}
    \item \mintinline{c}{atoll}: returns \mintinline{c}{long long int}
    \item \mintinline{c}{strtoll}: returns \mintinline{c}{long long int}
    \item \mintinline{c}{strtoul}: returns \mintinline{c}{unsigned long int}
    \item \mintinline{c}{strtoull}: returns \mintinline{c}{unsigned long long int}
\end{itemize}

\subsubsection*{Conclusion}
\mintinline{c}{strtoul} is chosen to be the function for integer arguments processing. This is
because:
\begin{enumerate}
    \item Its return type is \mintinline{c}{unsigned long int}, which has the same size and range as
    \mintinline{c}{unsigned int} \cite{basic_types_storage}. In comparison with a
    \mintinline{c}{long long} type, it would use less computational resources and hence comply with
    \textbf{Programming Practice 3}.
    \item It stops reading the string at the first character it cannot recognize as part of a number
    \cite{strtoul}. This is helpful for error checking the string.
\end{enumerate}


\subsubsection{Operation Mode (M)}
There are only two valid enum values for the operation mode: \mintinline{c}{CPU} and
\mintinline{c}{OPENMP}. The program uses \mintinline{c}{strncmp()} to compare the argument
\textbf{M} with the strings \mintinline{c}{"CPU"} and \mintinline{c}{"OPENMP"}, and assign the
operation mode to the corresponding enum value.

The program will return an error message and exit if the argument \textbf{M} is other value.

\subsubsection{Input File}
The file name is stored into \mintinline{c}{char *} variable and the file content is processed in
N-body data initialisation stage later.

The program will return an error message if the option \mintinline{bash}{-f} is specified but no
file name is provided.

\subsection{N-Body Data Initialisation}
\begin{listing}[ht]
\begin{minted}[frame=lines, framesep=2mm]{c}
#define DEFAULT_X ((float)rand() / RAND_MAX)
#define DEFAULT_Y ((float)rand() / RAND_MAX)
#define DEFAULT_VX 0
#define DEFAULT_VY 0
#define DEFAULT_M (1 / (float)N)
\end{minted}
\caption{Default value of each n-body member variable} \label{listing:default}
\end{listing}

Macros are defined for the default value of each n-body member variable. They will be used for
generating random data and setting the unspecified variables in an input file.

\subsubsection{Generating Random Data}
Generating random data is straightforward. The value of each member variable is set to the
corresponding macro.

\subsubsection{Processing Input File}
\begin{tcolorbox}
\textbf{Problem:} The input file contains lines of comma separated values, each representing the
initial states of an n-body. The input file may also contain comment lines.

\bigskip\textbf{Goal:} Read the input file and process only non-comment lines. Default values should
be assigned for unspecified variables.
\end{tcolorbox}

\subsubsection*{Step 1: Reading the lines}
Each line in the input file can be categorised as follows:
\begin{enumerate}
    \item \textbf{Category 1:} The line is a comment line
    \item \textbf{Category 2:} The line is an empty line
    \item \textbf{Category 3:} The line do not have the correct number of commas
    \item \textbf{Category 4:} The line contains 5 values and the correct number of commas
    \item \textbf{Category 5:} The line contains unspecified values and the correct number of commas
\end{enumerate}

Lines that belong to either \textbf{Category 4} or \textbf{Category 5} should only be read and
processed by the program. This is implemented by modifying the \mintinline{c}{readLine()} function
provided in the solution of Lab 1 Exercise 5, as shown in
\hyperref[listing:readline]{\textbf{Listing \ref*{listing:readline}}}.

\begin{listing}[ht]
\begin{minted}[linenos, frame=lines, framesep=2mm, fontsize=\footnotesize]{c}
    while ((c = (char)getc(f)) != EOF) {
        // Case 1: ignore any line starting with '#'
        // Case 2: ignore any blank line
        if (i == 0 && (c == '#' || c == '\n')) {
            while (c != '\n') {
                c = (char)getc(f);
            }
        } else {
            ...
\end{minted}
\caption{Code snippet of \mintinline{c}{read_line()} function.} \label{listing:readline}
\end{listing}

For \textbf{Category 3}, a function is created to ensure that only 4 commas appear in a non-comment
line. The function will exit the program with an error message if the line does not contain the
correct number of commas.

\subsubsection*{Step 2: Extracting the Values}
Now each line will have exactly 4 commas. Some lines will have all 5 values and some of them will
have unspecified values. Unspecifed values will then be assigned with the default values defined in
\hyperref[listing:default]{\textbf{Listing \ref*{listing:default}}}. Firstly, the line has to be
split into a series of tokens using comma as delimiter. Each token can then be converted into
\mintinline{c}{float} using the functions provided in C standard library.

\subsubsection*{Iteration 1}
The first attempt to split the line into tokens was using the function \mintinline{c}{strtok} in C
\mintinline{c}{string.h} library. It is works well when the line is in correct format, and there
must be a space after each comma. However, in the case where the line has consecutive commas without
space between them (e.g. \mintinline{c}{0.1f,,0.2f,,}), \mintinline{c}{strtok} would extract the
second token as \mintinline{c}{0.3f} instead of \mintinline{c}{NULL}.

After further experiments, it is confirmed that \mintinline{c}{strtok} is unable to recognise empty
tokens. Therefore, a different approach has to be adapted.

\subsubsection*{Iteration 2}
\begin{listing}[ht]
\begin{minted}[linenos, frame=lines, framesep=2mm, fontsize=\footnotesize]{c}
static char *tokenise(char *buffer) {
    static char *buffer_start = NULL;

    if (buffer != NULL) buffer_start = buffer;

    // see if we have reached the end of the line
    if (buffer_start == NULL || *buffer_start == '\0') return NULL;

    // return the number of characters that are not delimiters
    const unsigned int n = strcspn(buffer_start, ",");

    // return token as NULL for consecutive delimiters
    if (n == 0) {
        buffer_start += 1;
        return NULL;
    }

    // save start of this token
    char *p = buffer_start;

    // bump past the delimiters
    buffer_start += n;

    // remove the delimiters
    if (*buffer_start != '\0') *buffer_start++ = '\0';

    return p;
}
\end{minted}
\caption{Code snippet of \mintinline{c}{tokenise()} function.} \label{listing:tokenise}
\end{listing}

The new approach attempts to address the problem in \textbf{Iteration 1} by creating a customised
version of \mintinline{c}{strtok} function (\hyperref[listing:tokenise]{\textbf{Listing
\ref*{listing:tokenise}}}). The lines of code that allows the \mintinline{c}{tokenise} function to
recognise empty tokens are line 13 - 16, where the function returns \mintinline{c}{NULL} for
consecutive delimiters case.

\subsection{Memory Allocation} \label{sec:malloc}
% use calloc instead of malloc+memset. TODO: prove by benchmark

As mentioned in \hyperref[subsec:args]{\textbf{Section \ref*{subsec:args}}}, memory error might
occur if the integer arguments \textbf{N} and \textbf{D} are large and the machine does not have the
sufficient memory. This can be avoided by having a \mintinline{c}{NULL} check on whether the memory
allocation has succeeded, as shown in \hyperref[listing:malloc]{\textbf{Lisitng
\ref*{listing:malloc}}}.

\begin{listing}[ht]
\begin{minted}[linenos, frame=lines, framesep=2mm, fontsize=\footnotesize]{c}
    nbodies = (nbody *)malloc(sizeof(nbody) * N);
    if (nbodies == NULL) {
        fprintf(stderr, "error: failed to allocate memory: nbodies\n");
        exit(EXIT_FAILURE);
    }

    activity_map = (float *)calloc(D * D, sizeof(float));
    if (activity_map == NULL) {
        fprintf(stderr, "error: failed to allocate memory: activity_map");
        exit(EXIT_FAILURE);
    }
\end{minted}
\caption{Dynamically allocating memory.} \label{listing:malloc}
\end{listing}

\subsubsection*{\mintinline{c}{malloc + memset} or \mintinline{c}{calloc} for \mintinline{c}{activity_map}?}
The variable \mintinline{c}{activity_map} needs to be read before it is assigned a value in the
\mintinline{c}{step()} function. Therefore, all bytes in the allocated storage must be initialise to
zero before \mintinline{c}{step()} function is executed. The C library function
\mintinline{c}{calloc} is suitable for this use case.

Several online sources \cite{malloc_vs_calloc_so, malloc_vs_calloc_blog, why_calloc} have suggested
that \mintinline{c}{calloc} is faster and uses less memory than \mintinline{c}{malloc + memset},
although the actual results might vary across different compilers. Since this will not be included
in the execution time of the simulation, thus it is assumed that the compiler would be able to
perform the appropriate optimisations.

\subsection{Testing}
\subsubsection{Number of Bodies (N)}
\renewcommand{\arraystretch}{1.3}
\begin{longtable}{|c|c|c|>{\columncolor[HTML]{B7E1CD}}c|}
  \hline \endfirsthead \rowcolor[HTML]{D8D8D8}
  Test Case & Argument & Expected Result & Status \\ \hline
  Negative number & $\textbf{N} = -123$ & Program exits with an error message & Pass \\ \hline
  Zero & $\textbf{N} = 0$ & Program exits with an error message & Pass \\ \hline
  Positive number & $\textbf{N} = 123$ & Program starts the simulation & Pass \\ \hline
  Exceeds \mintinline{c}{UINT_MAX} & $\textbf{N} > \mintinline{c}{UINT_MAX}$ & Program exits with
  an error message & Pass \\ \hline
  \caption{Test cases for the argument \textbf{N}.}
\end{longtable}
\renewcommand{\arraystretch}{1}

\subsubsection{Activity Grid Dimension (D)}
\textcolor{red}{N.B.} See \hyperref[subsec:maxd]{\textbf{Section \ref*{subsec:maxd}}} for the
justification about the test case $\textbf{D} > 46430$.

\renewcommand{\arraystretch}{1.3}
\begin{longtable}{|c|c|c|>{\columncolor[HTML]{B7E1CD}}c|}
  \hline \endfirsthead \rowcolor[HTML]{D8D8D8}
  Test Case & Argument & Expected Result & Status \\ \hline
  Negative number & $\textbf{D} = -123$ & Program exits with an error message & Pass \\ \hline
  Zero & $\textbf{D} = 0$ & Program exits with an error message & Pass \\ \hline
  Positive number & $\textbf{D} = 123$ & Program starts the simulation & Pass \\ \hline
  Exceeds 46430 & $\textbf{D} > 46430$ & Program exits with an error message & Pass \\ \hline
  \caption{Test cases for the argument \textbf{D}.}
\end{longtable}
\renewcommand{\arraystretch}{1}

\subsubsection{Operation Mode (M)}
\renewcommand{\arraystretch}{1.3}
\begin{longtable}{|c|c|c|>{\columncolor[HTML]{B7E1CD}}c|}
  \hline \endfirsthead \rowcolor[HTML]{D8D8D8}
  Test Case & Argument & Expected Result & Status \\ \hline
  CPU mode & $\textbf{M} =$ CPU & Simulation starts starts in CPU mode & Pass \\ \hline
  OPENMP mode & $\textbf{M} =$ OPENMP & Simulation starts in OPENMP mode & Pass \\ \hline
  Other inputs & $\textbf{M} = 123,$ abc, ... & Program exits with an error message & Pass \\ \hline
  \caption{Test cases for the argument \textbf{M}.}
\end{longtable}
\renewcommand{\arraystretch}{1}

\subsubsection{Number of Simulation Iterations (-i I)}
\renewcommand{\arraystretch}{1.3}
\begin{longtable}{|c|c|c|>{\columncolor[HTML]{B7E1CD}}c|}
  \hline \endfirsthead \rowcolor[HTML]{D8D8D8}
  Test Case & Argument & Expected Result & Status \\ \hline
  Negative number & $\textbf{I} = -123$ & Program exits with an error message & Pass \\ \hline
  Zero & $\textbf{I} = 0$ & Simulation starts in visualisation mode & Pass \\ \hline
  Positive number & $\textbf{I} = 123$ & Simulation starts in iteration mode & Pass \\ \hline
  Exceeds \mintinline{c}{UINT_MAX} & $\textbf{I} > \mintinline{c}{UINT_MAX}$ & Program exits with
  an error message & Pass \\ \hline
  \caption{Test cases for the argument \textbf{I}.}
\end{longtable}
\renewcommand{\arraystretch}{1}

\subsubsection{Input File (-f F)}
\renewcommand{\arraystretch}{1.3}
\begin{longtable}{|c|c|>{\columncolor[HTML]{B7E1CD}}c|}
  \hline \endfirsthead \rowcolor[HTML]{D8D8D8}
  Test Case (file) & Expected Result & Status \\ \hline
  File name exists in system & Program starts to parse the file & Pass \\ \hline
  File name does not exist in system & Program exits with an error message & Pass \\ \hline
  \caption{Test cases for the argument \textbf{F}.}t
\end{longtable}
\renewcommand{\arraystretch}{1}

\renewcommand{\arraystretch}{1.3}
\begin{longtable}{|c|c|>{\columncolor[HTML]{B7E1CD}}c|}
  \hline \endfirsthead \rowcolor[HTML]{D8D8D8}
  Test Case (line in file) & Expected Result & Status \\ \hline
  Blank line & Program exits with an error message & Pass \\ \hline
  Comment line & Program exits with an error message & Pass \\ \hline
  Non-comment line, incorrect format & Program exits with an error message & Pass \\ \hline
  Non-comment line, correct format & Values are extracted and stored  & Pass \\ \hline
  \caption{Test cases for the lines in the input file.}
\end{longtable}
\renewcommand{\arraystretch}{1}

\subsection{Changes Introduced in Second Iteration}
Below is the list of changes introduced during the iterative implementation process.

\subsubsection{Maximum Value of Activity Grid Dimension (D)} \label{subsec:maxd}
\begin{listing}[ht]
\begin{minted}[linenos, frame=lines, framesep=2mm, fontsize=\small]{c}
    /* Loop through the `activity_map` to normalise the body counts */
    const int n = (int)(D * D);
#pragma omp parallel for ... if (M == OPENMP)
    for (i = 0; i < n; i++) {
        activity_map[i] *= (float)D / (float)N;
    }
\end{minted}
\caption{Code snippet of \mintinline{c}{step()} function.} \label{listing:step}
\end{listing}

\noindent Two problems were identified in \hyperref[listing:step]{\textbf{Listing
\ref*{listing:step}}}:
\begin{enumerate}
    \item A compile error C3016 was encountered while attempting to parallelise the code. This is
    due to the index variable in OpenMP \mintinline{c}{for} statement must have signed integral type
    \cite{c3016}. This can be fixed by changing the type of variable \mintinline{c}{i} and
    \mintinline{c}{n} to \mintinline{c}{int}.
    \item In line 2, arithmetic overflow will occur when \mintinline{c}{D} is larger than 46341 as
    $46341^2 > 2^{31} - 1$. This can be fixed by setting the maximum possible value of \mintinline{c}{D} to
    46340.
\end{enumerate}

\textcolor{red}{N.B.} To prevent the compile error C3016 from occurring for argument \textbf{N} and
\textbf{I}, the maximum possible value for \textbf{N} and \textbf{I} are changed from
\mintinline{c}{UINT_MAX} to \mintinline{c}{INT_MAX}. (changes are applied to \textbf{Category 4} in
\hyperref[subsec:args]{\textbf{Section \ref*{subsec:args}}})

\section{Serial Implementation}
\subsection{Compute Bound Optimisations}
With the programming practices defined (\hyperref[sec:practices]{\textbf{Section
\ref*{sec:practices}}}) in mind, the whole implementation process is simplified as good optimisation
opportunities are more likely to be identified in the first iteration.

\subsubsection{Reduced Arithmetic Operations}

\subsubsection{Reduced Function Calls}

\subsection{Benchmark Configuration}
To avoid having inconsistency in the benchmark results, all benchmarks are performed under the same
configuration.

\subsubsection{Hardware Specifications}
\begin{enumerate}
    \item Operating system: Windows 10 64-bit
    \item CPU: Intel\textsuperscript{\textregistered} Core\texttrademark\ i7-7700HQ, 4 cores, 8
    threads
    \item RAM: 16GB DDR4-2400MHz
\end{enumerate}

\subsubsection{Number of Runs per Benchmark}
\textbf{Definition:} A single benchmark is a set of unique input argument combinations.

\noindent It is unlikely that the result of every same benchmark run is the same. There will be
minimal execution time difference between each run due to uncontrollable environment variables. This
can be mitigated by taking the average execution time of multiple benchmark runs to minimise the
error rate and randomness. Each benchmark result would be more accurate as well when compared to a
single benchmark run.

\textbf{Conclusion:} It is decided to calculate the average execution time of a single benchmark
over 10 runs.

\subsection{Benchmark Results}
\begin{table}[H]
\centering
\begin{tabular}{|c|c|c|}
\hline
\rowcolor[HTML]{D8D8D8}
N    & Average Execution Time of 10 runs (seconds) \\ \hline
256  & 0.325403                                    \\
512  & 1.294606                                    \\
1024 & 5.187437                                    \\
2048 & 21.057497                                   \\ \hline
\end{tabular}
\end{table}


\section{Parallel Implementation}

% omp_set_nested? does it work?

\subsection{Benchmark Results}

\printbibliography

\end{document}
